\chapter{Introduction}
\thispagestyle{empty}
In many fluid systems, flow transition phenomena occur due to a change in a \emph{control} parameter, where the fluid motion is self-organized in patterns which depend on the symmetry in the system. The changes in the control parameter can lead to instability which can drive the system out of one equilibrium state and into another, eventually leading to chaos. It is of interest to understand these critical conditions under which transitions occur, and many classical examples which exhibit such transitions, have been studied extensively.
%In \emph{Poiseuille flow}, transition  from laminar to turbulent flow in a circular pipe arises due to a critical speed in the center of the pipe \cite{Pois_tran,Poi_sub_tran}.
With infinite parallel plates heated from below, the transition from pure conduction to  \emph{Rayleigh-B\'enard} convection occurs at a critical temperature difference between the plates.
Another classical example is the differentially heated co-rotating cylinders, which has been employed to model {large-scale} geophysical fluids ~\cite{Lew_Henn,Lew_DoubleHopf}. The fluid flow between the cylinders undergoes successive transitions from axisysmmetric flow to steady waves, to amplitude vacillating waves, to more irregular flows  upon increasing the temperature difference between the rotating boundaries and/or the rotation rate of the cylinders.

In this thesis, the fluid system of interest is sheared annular electroconvection. This situation is analogous to a {two-dimensional} version of the  differentially heated co-rotating cylinder system, where the transport of charge and the electric body force resulting from an applied electric potential are analogous to the transport of heat and the buoyancy force due to the temperature differences, respectively. The crucial difference is in the choice of fluid, i.e. a liquid crystal in {Smectic A} phase, which essentially can be considered a {2-D} fluid. It is of intrinsic interest to study this system which is capable of providing insight into transition phenomena in buoyancy driven rotating flows.

To obtain a better understanding of the transitions of the flow in sheared annular electroconvection, experimental and numerical methods have previously been undertaken.
Experimental results have been reported by Morris \& de Bruyn~\cite{mao1995,morris1992,experi_elec} beginning with the visualization of \emph{Rayleigh-B\'enard} convective cells in rectangular geometry. In the case of annular flow, results have been generated with a laboratory setup consisting of a thin liquid crystal film suspended between two annular electrodes with the inner anode rotating at a constant rate. In this experiment, control parameters such as the potential difference, the distance between the two electrodes, and the angular rate of the inner electrode can be varied. Under small electric potential difference, the system exhibits axisymmetric flow, in which the velocity of the fluid is in the azimuthal direction and the charge diffuses between the boundaries. At a critical potential difference, a primary transition occurs to rotating waves in which the charge is convected by the flow. Beyond this potential difference, it is observed that the system  undergoes successive transitions from the rotating wave state, to amplitude vacillating waves, to irregular flow.

%Smectic electroconvection begins with the experimental work of Morris \cite{mao1995,morris1992,experi_elec} and the visualization of \emph{Rayleigh-B\'enard} convective cells in rectangular geometry.

Using the experimental results as a backdrop, the sheared annular electroconvection system was modelled with a system of partial differential equations (PDEs) and algebraic equations (AEs) which were subsequently analyzed~\cite{EDCTAFUCF,BAEWIS,AEWSDaDe,linearstability,WNAEISFF,Annular}.
%developed the analytical base solution and performed an analysis of the resulting partial differential equations (PDEs) and algebraic equations (AEs) that model the annular electroconvection.
Up to this point in the literature, the focus has been on the primary flow transition from the axisymmetric base state to the rotating waves.
Subsequently Tsai and coworkers~\cite{PeiChunTsain,DNSSAE,tsai2004aspect,LSSE-PeDa} developed a numerical solver for the full model. In that work, the model equations are  integrated forward in time and the evolution of the physical quantities is observed for a small set of initial conditions varying values of the control parameters. We refer to such methods collectively as numerical experimentation.  Using this method, Tsai~et~al.~\cite{PeiChunTsain,DNSSAE} observed that the axisymmetric flow undergoes a sequence of transitions as the Rayleigh number is increased, eventually resulting in a chaotic regime. Tsai's observations were consistent with earlier analysis that predicted that the primary transition is a supercritical Hopf bifurcation~\cite{linearstability}. However, numerical experimentation is not able to compute unstable flows, nor is it able to accurately determine transition points, or determine the type of bifurcation associated with the transition. Therefore, due to the limitations of the method, the nature of the transitions could not be verified.

Alternative methods can be used to compute the asymptotic behaviour of solutions as parameters of the model are changed.    Possible asymptotic states of the flow may include steady states, periodic orbits, {quasi-periodic}, invariant tori or other more complex invariant limit sets. Finding the transition due to critical conditions can be rephrased in the language of dynamical systems theory as finding the critical parameter values at which bifurcations in the flow state occur. Practical methods for computing the transitions are known collectively as numerical bifurcation methods (NBMs) which consist of continuation methods that identify certain asymptotic sates and a linear stability analysis that investigates their local behaviour. The NBMs have advantages over the numerical experimentation method due to the systematic and unambiguous computation of the equilibria and bifurcations. They are able to compute unstable solutions as well as solutions exhibiting {bi-stability} without knowledge of the initial conditions that lead to these solutions, and thus are able to provide a clearer picture of the dynamics of the system. Moreover, they provide insight into the underlying physics that governs transition in the flow. However, the advantages of this approach is balanced by the sophisticated linear algebra required in its implementation.

In this thesis, we present a numerical bifurcation analysis of the sheared annular electroconvection problem. In particular, a matrix-free method based on numerical integration of the model equations is implemented to compute steady states and limit cycles, which correspond to the axisymmetric flow and rotating waves, respectively. In the matrix-free approach, these solutions can be computed as fixed points of two different discrete dynamical systems. This is possible in the case of rotating waves because we take into account that they are a special type of limit cycle. In particular, they are waves of constant amplitude that rotate at a constant phase speed.  The development and implementation of the matrix-free method for the computation of rotating waves represents the main original contribution of this thesis. With this method, we resolve the secondary transition in the sheared annular electroconvection problem, complimenting the work of Tsai~et~al.~\cite{EDCTAFUCF,BAEWIS,AEWSDaDe,linearstability,WNAEISFF,Annular}.

In Chapter 2, we provide details of the electroconvection experiment and of the mathematical and numerical models used to study it. We also discuss previous numerical and experimental results. In Chapter 3, we introduce the matrix-free methods used in the thesis, and discuss some details of the implementation. These methods provide an advantage over the more common methods used in various software packages such as MATCONT and AUTO~\cite{MATCONT}, in particular, by framing the computation in a way that does not require the storage of large matrices. We also expand upon these advantages in Chapter 3. The results and a comparison to previous work are presented in Chapter 4. Finally, the thesis closes with some reflections on the progress that has been made with an eye towards future work utilizing numerical bifurcation methods for computing more complex flows.

%The aim of this thesis is to give an overview of the implemented matrix-free numerical bifurcation methods; in particular a different approach is carried out to compute limit cycles using integration-time map, where the rotating waves are computed directly for general periodic orbit discussed in Chapter 3. The advantages of this approach over the standard approach that is used in different packages such as MATCONT and AUTO ~\cite{MATCONT} arise in solving a much more reduced system as well as the use of matrix-free algorithms. The implemented methods are tested on the sheared annular electroconvection and this is developed in Chapter 2. Finally, Chapter 4 presents the results achieved through-out the research with a comparison to previous results. The main contribution is a new approach developed to compute the flow of rotating waves. In the case of the electroconvection flow, this techniques was able to resolve the secondary transition extending the work of Tsai.
%The thesis closes with some reflections on the progress that have been made with an eye towards future work utilizing numerical bifurcation methods with more complex flow.

%\section{Contributions}
%In this thesis, the initial value solver of the perturbation equations from the base state of the sheared annular electroconvection implemented by (Tsai \cite{PeiChunTsain}) is modified and tailored to the computation needed and is used as a black box in our matrix-free numerical bifurcation method. This method is implemented in MATLAB to trace two type of invariant limit sets (steady states and limit cycles) of a dynamical system. The results of our computation of steady states and the primary bifurcation is compared and validated with the previous experimental \cite{experi_elec}, analytical \cite{linearstability} and numerical results \cite{EDCTAFUCF,SBSAE,DNSSAE} performed.
%A different approach is implemented to compute limit cycles corresponding to the flow of rotating waves \cite{rel_equ}, that is, the phase speed of the rotating wave is computed instead of the period. Using this technique, the degree of freedom are reduced extensively from the approach that requires to compute the period. The solution curve corresponding to the flow of rotating waves is traced and the secondary bifurcation point is determined for the first time using linear stability analysis.



%
%This thesis develops a matrix-free bifurcation method which is able to classify two types of invariant limit sets, (steady states and limit cycles). Using the newly developed method, we can compute the asymptotic behaviours of the equilibrium solution as parameters of the model are changed.
