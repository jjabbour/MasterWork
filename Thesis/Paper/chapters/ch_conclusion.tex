\chapter{Conclusion}
Numerical bifurcation methods are powerful tools that utilize sophisticated linear algebra techniques that, when carried out successfully, can be computationally efficient.  The main result of this thesis is the use of these techniques on a model for  two-dimensional sheared annular electroconvection and the generation of a bifurcation diagram where the parameter space is divided into regions of topologically different behaviours together with their phase portrait.
These methods have advantages over numerical experimentation because they are not limited to finding stable invariant limit sets. Quite often when analyzing complex systems and in particular with the system studied in this thesis, more unstable solutions are found than stable ones and these techniques allow both to be utilized to understand the important structure of all solutions and how they interrelate.

In this thesis a matrix-free numerical bifurcation method was implemented in MATLAB that traces two types of compact invariant limit sets of the dynamical system under study. In particular, steady states and limit cycles have been studied. The method consists of a natural continuation method, that computes and traces solution curves of the respective limit sets in the parameter space, with a linear stability analysis that identifies the local behaviour of these sets. This method has been applied to identify and classify different solution regimes and flow transitions in sheared annular electroconvection, in which steady state solutions correspond to axisymmetric flow and periodic solutions correspond to rotating wave flow. The main parameter of interest in this work is the Rayleigh number, $\mathcal{R}$, and by gradually increasing its value, the flow is shown to undergo successive transitions from axisymmetric, to rotating waves then to amplitude vacillating waves. The flow of rotating waves is computed using an approach distinct from the standard one by solving for the phase speed of the wave instead of the traditional technique of solving for its period. This {matrix-free} method greatly reduces the number of degrees of freedom of the system from what is required with the standard technique implemented, e.g. in the software package AUTO.

The solution curve corresponding to axisymmetric flow is traced by varying the Rayleigh number $\mathcal{R}$ while fixing all the other model parameters. The primary transition is detected at $\mathcal{R}_c = 572$ and is identified as a supercritical Neimark-Sacker of the flow map where the flow transitions to a state of rotating waves. Validation of this observation is made with previous experimental, theoretical and simulation results. We also traced the solution curve corresponding to flow of rotating waves for the same fixed model parameters and we were able to pinpoint, for the first time, the secondary transition at $\mathcal{R}_{c_2} = 636$. This bifurcation is also identified as a Neimark-Sacker type of the corresponding flow map. Beyond this transition, the flow transitions to amplitude vacillating waves and this was identified by time integrating the solution corresponding to the rotating waves with a small perturbation.  The growth of the amplitude of the vacillation was computed for successive $\mathcal{R}$ beyond this bifurcation and based on this preliminary work we hypothesize that the secondary bifurcation is also supercritical, because the amplitude of the vacillation grows monotonically with $\mathcal{R}$ from zero. To confirm whether this bifurcation is either supercritical or subcritical, the computation of the normal form coefficient of the flow map is required.
%This implementation could be extended to trace the invariant limit set (torus) corresponding to the flow of amplitude vacillating wave.
%\section{Future Work}

This thesis represents the first step in an extensive bifurcation analysis of sheared annular electroconvection that will help develop an understanding of this complex and interesting system.  We have implemented {matrix-free} methods for the continuation of axisymmetric and rotating waves, and have shown the feasibility and validity of their application to this problem.  Although this required extensive work, we are now in a position to easily extend the study to a large variety of phenomena.  In particular, with trivial modification, we can explore the effects of other nondimensional parameters (e.g. the Reynolds number $\mathrm{Re}$, the aspect ration $\alpha$, and the Prandtl number $\mathcal{P}$).  We may also study a variety of flows found at various locations of parameter space.  Of particular interest are rotating waves consisting of isolated or elongated vortices, because their origin and nature are not known.   Because these flows are rotating waves, it may be possible to use the current implementation without modification, although it may be necessary to extend the code to incorporate pseudoarclength continuation. Also, with only slight modification, the code could be used to continue the {Neimark-Sacker} bifurcations in more than one parameter, enabling us to trace out a detailed picture of the dynamics of the system.

It would also be of great interest to adapt our method to include the possibility of tracing out the solutions corresponding to the amplitude vacillating waves. This, however, will involve more than a simple extension of the code.  Indeed, an additional condition will have to be found which will determine an additional unknown quantity, e.g. the period of vacillation.  Alternatively, it may be possible to use the {Newton-Krylov} method of Sanchez~et~al.~\cite{sanchez2010computation}, which traces invariant {2-tori}.  This thesis has set up the possibility, and the motivation, to embark on this important and ground-breaking work, which will be a primary thrust of future work.



%The implemented numerical bifurcation method traces the two types of solutions of the sheared annular electroconvection (axisymmetric and rotating wave). Using this implementation, we could extended this work to study the effect of the other nondimensional parameters $(\mathrm{Re},\alpha,\mathcal{P})$ on the primary transition validating previous results on numerous codimension two point obtained by Morris et al \cite{BAEWIS,PeiChunTsain}. We also have knowledge from previous physical and numerical experimentation that for $\mathcal{R}>\mathcal{R}_{c_2}$, the flow is an amplitude vacillating wave and at $\mathcal{R}_{c_3}$, this flow transitions to another state of rotating waves with different mode from the rotating wave that is obtained above the primary transition and was computed in this thesis. This approach can be uses to trace this solution branch.
%
%At this point, the primary thrust of future work would be to extend this technique to trace the solution curve corresponding to amplitude vacillating waves. It might be possible to extend the approach of the rotating wave, by solving a closed system of equations, where the period of the vacillation as an unknown. In the work of Sanchez~et~al~\cite{sanchez2010computation}, a {Newton-Krylov} method is presented for tracing invariant-2-tori. This method was tested on the 2D thermal convection in rectangular geometry which can be seen analogous to the sheared annular electroconvection. This could be the first step in the direction of tracing the curve for the amplitude vacillating wave.
